Im Rahmen eines allgemein psychologischen Experimentes, wird ein Serious Game entwickelt. Dabei wird ein Experiment, welches die menschliche Aufmerksamkeit auf visuelle Eindrücke prüft, mit dem Fokus auf die Beurteilung der zeitlichen Reihenfolge, als Kern für ein Serious Game genutzt. Das bedeutet, der Kern des Serious Games entspricht dem klassischen Aufbaus eines entsprechenden Experiments. Nach erfolgreichem Absolvieren des Serious Games entsteht ein Datensatz der den durchführenden Psychologen Ergebnisse liefert, die mit den Ergebnissen des Ursprungsexperiment vergleichbar sind. Das Serious Game wird in der Praxis getestet und die Ergebnisse werden anschließend mit denen von konventionellen Experimenten, die nicht als Serious Game präsentiert wurden, verglichen. Zusätzlich wird während des Tests die Motivation der Versuchsperson ermittelt, um beurteilen zu können, ob das Serious Game die gleichen Ergebnisse bei gesteigerter Motivation liefert.

