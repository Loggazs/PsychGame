
Wenn von Serious Games gesprochen wird, sind in erster Linie Videospiele gemeint, die Lerninhalte vermitteln wollen \cite{susi2007serious}. Durch das Einbetten der Lerninhalte  in die spielerischen Elemente von Videospielen findet ein Wissenstransfer bei gesteigerter Motivation statt. Grundkonzepte von Spielen lassen sich dabei hervorragend auf den Lernprozess übertragen. Zum Beispiel weisen Videospiele eine Lernkurve auf, die mit zunehmender Spieldauer immer weiter ansteigt. Der Spieler lernt zuerst die grundlegenden Mechaniken des Spiels, wie z.B bei Super Mario das Laufen und das Springen. Nach einer kleinen Eingewöhnungsphase wird auf dem bereits angelernten Wissen des Spielers aufgebaut. Immer mehr Mechaniken und Probleme werden präsentiert an denen der Spieler seine Fähigkeiten weiter trainieren und verbessern kann. Es gilt herauszufinden, ob sich solche Konzepte aus Videospielen auch auf Experimente in der Psychologie übertragen lassen. Diese Experimente werden meist als relativ lang und eintönig wahrgenommen. Das psychologische Experiment als Serious Game wird Versuchspersonen mit erhöhter Motivation dazu anregen das Experiment bis zum Ende durchzuführen und dabei auch bis zum letzten Durchlauf noch konzentriert zu sein. Die Parameter des Experiments müssen eingehalten werden, denn die Versuchsergebnisse des Serious Games müssen für die Forschung weiter verwendbar sein.
Es gilt herauszufinden wie viel Videospielelemente in das psychologische Experiment einfließen können, ohne dass dies Ergebnisse verfälscht. Gleichzeitig wird evaluiert wie viel psychologisches Experiment in dem Serious Game enthalten ist, ohne das es wieder lang und eintönig wirkt.


