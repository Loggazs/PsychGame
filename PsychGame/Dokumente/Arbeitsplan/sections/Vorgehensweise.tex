
Abbildung~\ref{fig:PhasenMeilensteinDiagramm} stellt das Vorgehen der Arbeit als Phasen-Meilenstein-Diagramm dar. Zu Beginn werden die Grundlagen gesammelt. Als Grundlage für das Experiment werden psychologische Grundlagen über Salienz \cite{Itti2007} und visuelle Suche \cite{Wolfe2008} benötigt. Mit Hilfe von Kenntnissen die in der Videospielentwicklung gewonnen wurden, wie unter anderem von Fullerton \cite{fullerton2014game} beschrieben, bietet dies die Grundlage für das Game Design des Serious Games.  Um das Serious Game umzusetzen, wird eine Spiele-Engine hinzugezogen, die Unreal Engine 4. Um am Ende das Ergebnis auswerten zu können, wird statistische Analyse benötigt. Kruschke \cite{kruschke2013bayesian} beschriebt wie man Datenmengen mit Hilfe von Bayes-Statistik vergleichen kann. Als Grundlage für die Bestimmung der Motivation währen der Tests, wird das Erfassen von Flow \cite{rheinberg2003erfassung} als Grundlage verwendet. \\
Es werden Interviews mit den Psychologen im Psylab, einem Labor an der Universität Paderborn in dem psychologische Experimente durchgeführt werden, geführt. In den Interviews werden die grundlegenden Parametern aus dem Ursprungsexperiment, wie es Krüger \cite{kruger2016fast} beschreibt, diskutiert. Im Gespräch wird ermittelt welche Parameter aus dem Experiment benötigt werden, welche der Parameter zu Gunsten des Serious Games werden können und welche fest sind. Entlang der Anforderungen wird ein Game Design erstellt. Das integrieren von Spielelementen in spiel-fremde Bereiche wird als Gamificaton bezeichnet\cite{deterding2011gamification}. Das Game Design bietet die Grundlage für das Serious Game. Es gibt vor wie die Abläufe im Spiel sind, welche Steuerung benutzt wird und durch welche Ästhetik das Spiel bestimmt wird. Mit dem Game Design als Grundlage wird dann in der Unreal Engine der Prototyp umgesetzt. Dafür werden Blueprints verwendet, die in der Unreal Engine genutzt werden, um Verhalten zu programmieren. Es müssen Assets erstellt werden und in die Engine eingepflegt werden.\\
Nach Fertigstellung des Prototypen wird dieser im Psylab getestet. Für die Durchführung der Tests wird ein Fragebogen erstellt. Dieser Fragebogen soll die Motivation der Testperson bestimmen. Die Testergebnisse werden nach der Durchführung gesammelt und mit Ergebnissen aus konventionellen Experimenten verglichen. Dazu wird auf statistische Analyse mit der Bayes Statistik zurückgegriffen.
Aus der Auswertung der Fragebögen und der statistischen Analyse kann dann der Erfolg des Serious Games ermittelt werden. 
\begin{figure}
\begin{tikzpicture} [
    auto,
    decision/.style = { diamond, draw=dsRed, thick, fill=dsRed,
                        text width=1em, text badly centered,
                        inner sep=0.00pt , font=\footnotesize},
    block/.style    = { rectangle, thick, draw=gray!50!black,
    				text width=12em, text centered,
                        rounded corners, minimum height=2em, font=\footnotesize},
    line/.style     = { draw, thick, ->, shorten >=2pt , font=\footnotesize},
  ]

%Knoten innerhalb einer Matrix

%Knotenbenennung Zahl = Zeile; a=links, b = mitte, c = rechts
  \matrix [column sep=2mm, 
		row sep=4mm
		] {
                    \node [text centered] () {\textbf{Phasen / Meilensteine}};  
                    	& \node [text centered] () {\textbf{Aufgaben / Methoden}};            
                    		& \node [text centered] () {\textbf{Resultate}};  \\
                   \node [block](1a) {Know-How aufbauen \"uber
					\begin{itemize} 
 					\item Unreal Engine
					\item Psychologische Grundlagen
					\item Flow/Motivation
					\item Statistische Analyse
					\end{itemize}};    
                   	          &                             
			&\\
		\node[text centered, minimum height=2mm](99){};
			&
			& \\
		\node [decision] (2a) {\textcolor{white}{1}};
			& 
				& \node [block,  draw=dsBlue, fill = dsBlue!20] 
				 		(2b) {Grundlagenwissen};   \\

		%%%%%%%%%%%%%%%%%%%%%%%%%%%%%%%%%%%%%%%%%%

		\node[block](7a){Experimentanalyse};
		           &                             
			&\\
		\node[text centered, minimum height=2mm](99){};
			&
			&\\
		\node [decision] (8a){\textcolor{white}{2}};
			&
				& \node [block, draw=dsBlue, fill = dsBlue!20]
				 		 (8b) {Anforderungskatalog};\\

		%%%%%%%%%%%%%%%%%%%%%%%%%%%%%%%%%%%%%%%%%%
		%\node[block](9a){Recherche bzgl. des Standes der Technik};
		%	&
		%		&\\
		%\node [decision] (10a){\textcolor{white}{5}};
		%	&
		%		& \node [block, draw=dsBlue, fill = dsBlue!20]
		%		 		(10b) {\"Uberblick \"uber den aktuellen Stand der Techniken};\\

		%%%%%%%%%%%%%%%%%%%%%%%%%%%%%%%%%%%%%%%%%%
		\node[block](11a){Konzept erarbeiten};
			&
				&\\
		\node[text centered, minimum height=2mm](99){};
			&
				&\\
		\node [decision] (12a){\textcolor{white}{3}};
			&
				& \node [block, draw=dsBlue, fill = dsBlue!20] 
				 		(12b) {Game Design};\\

		%%%%%%%%%%%%%%%%%%%%%%%%%%%%%%%%%%%%%%%%%%
		\node[block](13a){Umsetzung};
			&
				&\\
		\node [decision] (14a){\textcolor{white}{4}};
			&
				& \node [block, draw=dsBlue, fill = dsBlue!20] 
				 		(14b) {Game Prototyp};\\

		%%%%%%%%%%%%%%%%%%%%%%%%%%%%%%%%%%%%%%%%%%
		\node[block](15a){Testen};
			&
				&\\
			\node[text centered, minimum height=2mm](99){};
			&
				& \\
		\node [decision] (16a){\textcolor{white}{5}};
			&
				& \node [block, draw=dsBlue, fill = dsBlue!20] 
				 		(16b) {Bewertung};\\
  };


%rote Pfeile (von Meilenstein zu Resultat)
  \begin{scope} [every path/.style={line, draw=dsRed}]

   \path (2a)  -- node [] {\parbox{4cm}{
					\begin{tabular}{ll}
					$\bullet$ &Fachliteratur \\
					$\bullet$ &Anwendung
					\end{tabular}
					}} (2b);

   \path (8a) -- node [] {\parbox{4cm}{
					\begin{tabular}{ll}
					$\bullet$ &Experimentparameter \\
					$\bullet$ &Experteninterviews
					\end{tabular}
					}} (8b);

% \path (10a) -- node [] {\parbox{4cm}{
%					\begin{tabular}{ll}
%					$\bullet$ &Fachliteratur
%					\end{tabular}
%					}} (10b);

   \path (12a) -- node [] {\parbox{4cm}{
					\begin{tabular}{ll}
					$\bullet$ &Fachliteratur\\ 
					$\bullet$ & Anforderungskatalog 
					\end{tabular}
					}} (12b);

   \path (14a) -- node [] {\parbox{4cm}{
					\begin{tabular}{ll}
					$\bullet$&Programmieren
					\end{tabular}
					}} (14b);

   \path (16a) -- node [] {\parbox{4cm}{
					\begin{tabular}{ll}
					$\bullet$&User Tests\\
					$\bullet$&Statistische Analyse
					\end{tabular}
					}} (16b);
  \end{scope}

%schwarze Pfeile (zwischen Meilensteinen)
  \begin{scope} [every path/.style={line, draw=black}]
   \path (1a) -- (2a);
   \path (2a) -- (7a);
   \path (7a) -- (8a);
   \path (8a) -- (11a);
  % \path (9a) -- (10a);
  % \path (10a) -- (11a);
   \path (11a) -- (12a);
   \path (12a) -- (13a);
   \path (13a) -- (14a);
   \path (14a) -- (15a);
   \path (15a) -- (16a);
  \end{scope}
\end{tikzpicture}
\caption{Phasen-Meilenstein-Diagramm}
\label{fig:PhasenMeilensteinDiagramm}
\end{figure}
\clearpage

%%%%%%%%%%%%%%%%%%%%%%