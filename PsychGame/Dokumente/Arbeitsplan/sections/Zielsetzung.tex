
In dieser Arbeit wird ermittelt, ob die Kombination aus Serious Game und allgemein psychologischen Experiment Ergebnisse liefert, die von Psychologen für ihre Forschung weiter verwendet werden können und gleichzeitig die Versuchspersonen motiviert und somit fokussiert hält. Es wird eine Experiment aus der Wahrnehmungspsychologie genommen, in dem die zeitliche Reihenfolge von Objekten mit salienten Reizen bestimmt werden soll. Dabei muss darauf geachtet werden, dass der Kern des Experiments mit seinen Grundparametern nicht verfälscht wird, damit die Versuchspersonen die gleichen Voraussetzungen im Serious Game haben, wie im konventionellen Experiment.
Das Serious Game wird auf zwei Aspekte überprüft. Zum einen, ob das Serious Game die gleichen Ergebnisse liefert wie die Experimente zuvor. Das soll bedeuten, dass die Ergebnisse, die das Serious Game liefert, vergleichbar sind mit denen eines herkömmlich durchgeführten Experiments. Um dies zu prüfen werden die Ergebnisse des Serios Games mit Bayes-Statistik mit den Ergebnissen von vorher durchgeführten konventionellen Experimenten verglichen. Ziel ist es eine möglichst genaue Übereinstimmung der Ergebnisse zu erlangen.
Zum anderen wird überprüft, ob die Versuchsperson das Serious Game als unterhaltsam empfindet. Der Versuchsperson wird ein Fragebogen vorgelegt, der Fragebogen orientiert sich dabei an bereits existierenden Fragebögen wie dem \grqq Fragebogen zur Erfassung aktueller Motivation in Lern- und Leistungssituationen" \cite{rheinberg2001fam}, um die Motivation der Versuchsperson zu ermitteln.
So wird ermittelt, ob das Serious Game für den Einsatz als psychologisches Experiment geeignet ist und gegebenenfalls einen Mehrwert liefert.
